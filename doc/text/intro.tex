\section{Introduction}

The aim of this project is providing an optimized hardware implementation for the
\textbf{AES Cryptosystem} targeting an FPGA device.

We design the hardware accelerator using \textbf{System Verilog} as an HDL. We use
\textbf{Vivado 2018.2} for the development, test and implementation of our project.
The target platform is a \textit{Xilinx Artix-7} FPGA (part no. XC7A100TCSG324C-1)
mounted on a \textit{Digilent Nexys4 DDR rev. C} board.

Since performance is the main driver of the project, we adopt an alternate implementation
of the AES cryptosystem based on \textit{Tboxes}. We squashed the trasformations
included in the original round structure, \texttt{SubBytes}, \texttt{ShiftRows},
\texttt{MixColumns} and \texttt{AddRoundKey} in a simpler one by mapping the AES
\textit{state bytes} through Tboxes. Tboxes are lookup tables that store precomputed
values for the \texttt{SubBytes} and \texttt{MixColumns} operations combined together.
Through fast Tbox \textit{lookups} and simple \textit{XOR} operations, it is possible
to complete a whole round of the cipher in a single clock cycle.

Another design criterion is providing support for all of the three different AES key sizes,
namely \textit{AES-128}, \textit{AES-192} and \textit{AES-256}.
Since the purpose of the project is producing an optimized AES core, we implement only
the \textit{ECB -- Electronic Codebook} mode, which encrypts
and decrypts each AES block separately.

The design is connected to the external world using a \textit{UART} interface.
The main features of the project are:
\begin{enumerate}
  \item support for \textit{AES-128}, \textit{AES-192}, \textit{AES-256} in \textit{ECB} mode
  \item Tbox-based implementation, completing a whole round in 1 clock cycle and
a whole encyption/decryption in 15 clock cycles, irrespectively of the key size
  \item unified encryption/decryption datapath, making the provided core easy to
replicate in the same design
  \item testing of the implementation on the final platform
\end{enumerate}
